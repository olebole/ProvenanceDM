In this document, we discuss a draft for an IVOA standard data model for describing the
provenance of data. We focus here on observational data, since provenance for
simulated data is already covered by the Simulation Data Model (SimDM 
\citep{std:SimDM}). However, the currently discussed version is abstract enough so that 
its core pattern could be applied to any kind of processes, including extraction of data from 
databases or even the flow of scientific proposals from application to 
acceptance and scheduling of the proposed observations. The provenance information 
could be used to check internal processes, e.g. if the proposal was approved by 
a person from a certain committee, if the time span between application and 
acceptance/refusal does not extend a certain period etc. 


\subsection{Goal of the provenance model}\label{sec:goals}
The goal of the provenance data model is to describe how provenance information 
can be modeled, stored and exchanged within the virtual observatory. Its scope 
is mainly to allow modelling of the flow of data, the relations between data 
and processing steps. Characteristics of observations like ambient conditions 
and instrument characteristics won't be modeled here explicitely. They can be 
included in the form of additional data linked to an observaton or as 
attributes to observation process.

In general, the model shall enable a scientist who has no prior knowledge about 
a data set to get more 
background information. This will help the scientist to decide if the data set 
is adequate for his research goal, judge its quality and get enough information 
to be able to trace back its history as far as possible. 

Provenance information can be recorded very detailed or in more general steps, 
depending on the desired level of detail of the individual projects that record 
provenance and the intended usage.



