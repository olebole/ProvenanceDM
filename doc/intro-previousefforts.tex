\subsection{Previous efforts}
The provenance concept was early introduced by the IVOA within the scope of the
Observation Data Model \citep[see IVOA note by ][]{note:observationdm}, as a
class describing where the data is coming from. A full observation data model
specifically dedicated to spectral data was then designed \citep[Spectral Data
Model,][]{std:SpectralDM}, as well as a fully generic characterisation
data model of the measurement axes of data \citep[Characterisation Data
Model,][]{std:CharacterisationDM}, while the progress on the provenance data
model was slowing down.

The IVOA Data Model Working Group first gathered various use cases coming from
different communities of observational astronomy (optical, radio, X-ray,
interferometry). Common motivations for a provenance tracing of their history
included: quality assessment, discovery of dataset progenitors, and access to
metadata necessary for reprocessing. The provenance data model was then designed
as the combination of \emph{Data processing}, \emph{Observing configuration},
and \emph{Observation ambient conditions} data model classes.
The \emph{Processing class} was embedding a sequence of processing stages which
were hooking specific ad hoc details and links to input and output datasets,
as well as processing step descriptions. Despite the attempts at
an UML description of the model and writing XML serialization examples, the IVOA
efforts failed to provide a workable solution: the scope was probably too
ambitious and the technical background too unstable. A compilation of these
early developments can be found on the IVOA site \citep{std:previousefforts}.
From 2013 onwards, the IVOA concentrated on use cases related to processing
description and decided to design the model by extending the basic W3C
provenance structure, as described in the current specification. 

Outside of the astronomical community, the Provenance Challenge series (2006 --
2010), a community effort to achieve inter-operability between different
representations of provenance in scientific workflows, resulted in the Open
Provenance Model (OPM) \citep{moreau2010}. Later, the W3C Provenance Working
Group was founded and released the W3C Provenance Data Model as Recommendation
in 2013 \citep{std:W3CProvDM}.  OPM was designed to be applicable to anything,
scientific data as well as cars or immaterial things like decisions. With the
W3C model, this becomes more focused on the web. Nevertheless, the core concepts
are still in principle the same in both models and are very general, so they
can be applied to astronomical datasets and workflows as well. The W3C model was
taken up by a larger number of applications and tools than OPM, we are therefore
basing our modeling efforts on the W3C Provenance Data Model, making it less
abstract and more specific, or extending it where necessary. 


The W3C model even already specifies PROV-DM Extensibility Points (section 6 in
\citealt{std:W3CProvDM}) for extending the core model. This allows one to
specify additional roles and types for each entity, agent or relation using the
attributes \texttt{prov:type} and \texttt{prov:role}. By specifying well-defined
values for the IVOA model, we can adjust the model to our needs while still
being compliant with W3C.

